\documentclass[a4paper, 12pts]{article}

\usepackage[top=3.5cm, bottom=3.5cm, left=3cm, right=3cm]{geometry}

\usepackage[T1]{fontenc}
\usepackage[utf8]{inputenc}
\usepackage[english]{babel}
\usepackage{textcomp}
\usepackage{listings}
\usepackage{authblk} %author tools
\usepackage{enumitem}
\usepackage{amssymb}
\usepackage{graphicx}
\usepackage[normalem]{ulem}

%\usepackage{hyperref} %pour les liens internet

%\usepackage{graphicx} %pour les images
\title{%
 TP4 C++ : Héritage et polymorphisme \\
 \large Cahier des charges détaillé}
\author{Edern HAUMONT}
\author{Théo THIBAULT}
\affil{B3133}
\date{\today}

%-----------------------------------------------------------------------------------------

\begin{document}

%\begin{titlepage}

\maketitle

%\end{titlepage}

%----------------------------------------------Title end

\section{Introduction}
	The developped software is a simple geometrical forms editor. It manages segments, convex polygones, rectangles. It also can create new objects by intersections and unions between other objects.
	Objects can be deleted and moved.\\
	The software will also have a "save and load" feature to add persistance to the applcation.

\section{Fonctionalities}
	\subsection{Objects}
		In all this document coordinates are entire numbers.
		An object has a name which is its unique identifier. It is only composed of letters and numbers\\
		Here is a a list of the objects that the application manages :
		\subsubsection{Segments}
			A segment is defined by its two extremty points.
		\subsubsection{Rectangle}
			A rectangle is defined by two points : the top left point and the bottom right point.
			All 4 segments which compose a rectangle are horizontal or vertical. Consequently, a rectangle cannot be oblic.
		\subsubsection{Polygone}
			A polygon is defined by a sorted list of points that are linked in order to form a closed figure.
			Our application only manages convec polygons,  that is that no diagonal is outside of the figure.
			A polygon is at least formed with 3 points.
	\subsection{Actions}
		\subsubsection{Add a segment}
			\uline{Command syntax} :
			S Name X1 Y1 X2 Y2
			\begin{itemize}
				\item Name : The identifier of the object.
				\item X1, Y1 : Coordinates of the first point.
				\item X2, Y2 : Coordinates of the second point.
			\end{itemize}
			\uline{Error cases}
			\begin{itemize}
				\item Error n°1 : Incorrect name.
				\item Error n°2 : Incorrect number of coordinates.
			\end{itemize}
			\uline{Output} : OK and confirmation or ERR and explanation.\\
			\uline{Note} : 
			Non entire coordinates will be truncated.\\
			You can give the same coordinates for the two points.

		\subsubsection{Add a rectangle}
			\uline{Command syntax} :
			R Name X1 Y1 X2 Y2
			\begin{itemize}
				\item Name : The identifier of the object.
				\item X1, Y1 : Coordinates of the top-left point.
				\item X2, Y2 : Coordinates of the bottom-right point.
			\end{itemize}
			\uline{Error cases}
			\begin{itemize}
				\item Error n°1 : Incorrect name.
				\item Error n°2 : Incorrect number of coordinates.
				\item Error n°3 : Incorrect point placement (Exemple, the second point is on the left of the first point)
			\end{itemize}
			\uline{Output} : OK and confirmation or ERR and explanation\\
			\uline{Note} : 
			Non entire coordinates will be truncated.\\
			You can give the same coordinates for the two points.

		\subsubsection{Add a convex polygon}
			\uline{Command syntax} :
			PC Name X1 Y1 X2 Y2 .. Xn Yn
			\begin{itemize}
				\item Name : The identifier of the object.
				\item Xi Yi : Coordinates of the ith point. Points must be given in cyclic order.
			\end{itemize}
			\uline{Error cases}
			\begin{itemize}
				\item Error n°1 : Incorrect name.
				\item Error n°2 : Incorrect number of coordinates (less than three points)
				\item Error n°3 : Non convex polygon.
				\item Error n°4 : Same coordinates used several times.
			\end{itemize}
			\uline{Output} : OK and confirmation or ERR and explanation\\
			\uline{Note} : 
			Non entire coordinates will be truncated.

		\subsubsection{Create an object by reunion of several other objects}
			\uline{Command syntax} :
			OR Name Name1 Name2 ... NameN
			\begin{itemize}
				\item Name : The identifier of the new object.
				\item NameI : The identifier of one of the included objets
			\end{itemize}
			\uline{Error cases}
			\begin{itemize}
				\item Error n°1 : Incorrect name.
				\item Error n°2 : Incorrect number of objects (less than 2).
				\item Error n°3 : Several uses of the same objects.
				\item Error n°4 : Inclusion of a non-existant object.
			\end{itemize}
			\uline{Output} : OK and confirmation or ERR and explanation\\
			\uline{Note} : 
			Non entire coordinates will be truncated.\\
			You can use objects already created by reunion or intersection.

		\subsubsection{Create an object by intersection of several other objects}
			\uline{Command syntax} :
			OI Name Name1 Name2 ... NameN
			\begin{itemize}
				\item Name : The identifier of the new object.
				\item NameI : The identifier of one of the intersected objets
			\end{itemize}
			\uline{Error cases}
			\begin{itemize}
				\item Error n°1 : Incorrect name.
				\item Error n°2 : Incorrect number of objects (less than 2).
				\item Error n°3 : Several uses of the same objects.
				\item Error n°4 : Intersection with a non-existant object.
			\end{itemize}
			\uline{Output} : OK and confirmation or ERR and explanation\\
			\uline{Note} : 
			Non entire coordinates will be truncated.\\
			You can use objects already created by reunion or intersection.\\
			You can intersect two unjoined objects.

		\subsubsection{Check a point-inclusion}
			\uline{Command syntax} :
			HIT Name X Y
			\begin{itemize}
				\item Name : The identifier of the tested object.
				\item X Y : Coordinates of the point that have to be into the object.
			\end{itemize}
			\uline{Error cases}
			\begin{itemize}
				\item Error n°1 : Incorrect or inexistant name.
				\item Error n°2 : Invalid format or number of coordinates.
			\end{itemize}
			\uline{Output} : YES or NO or ERR and explanation\\
			\uline{Note} : 
			Non entire coordinates will be truncated.

		\subsubsection{Delete an object}
			\uline{Command syntax} :
			DELETE Name1 Name2 ... NameN
			\begin{itemize}
				\item NameI : One of the objects to delete.
			\end{itemize}
			\uline{Error cases}
			\begin{itemize}
				\item Error n°1 : One of the names were invalid or non-existant. The operation is aborted.
			\end{itemize}
			\uline{Output} : OK or ERR and explanation\\
			\uline{Note} : 
			If one of the names is incorrect, no object is deleted.

		\subsubsection{Move an object}
			\uline{Command syntax} :
			MOVE Name dX dY
			\begin{itemize}
				\item Name : Identifier of the object to move.
				\item dX dY : Coordinates of the moving vector.
			\end{itemize}
			\uline{Error cases}
			\begin{itemize}
				\item Error n°1 : Incorrect or inexistant name.
				\item Error n°2 : Invalid format or number of coordinates.
			\end{itemize}
			\uline{Output} : OK or ERR and explanation\\
			\uline{Note} : 
			Non entire coordinates will be truncated.

		\subsubsection{Enumeration}
			\uline{Command syntax} :
			LIST\\
			\uline{Error cases} : No error case.\\
			\uline{Output} : Give a list of the objects (Identifier and type). One object by line.\\
			\uline{Note} : 
			If there is no object, nothing is returned.\\
			The different types of objects are : Segment, Rectangle, Polygon, Intersection and Reunion.

		\subsubsection{Undo the last action}
			\uline{Command syntax} :
			UNDO\\
			\uline{Error cases}
			\begin{itemize}
				\item Error n°1 : No action to undo.
			\end{itemize}
			\uline{Output} : OK and description of the undid command or ERR and explanation\\
			\uline{Note} : 
			The action-hictoric has a size of 20 actions.

		\subsubsection{Redo the last undo}
			\uline{Command syntax} :
			REDO\\
			\uline{Error cases}
			\begin{itemize}
				\item Error n°1 : No undo to redo.
			\end{itemize}
			\uline{Output} : OK and description of the redid command or ERR and explanation\\
			\uline{Note} : 
			You cannot redo an action if you did something beetween the undo and your redo tentative.

		\subsubsection{Load an old draw}
			\uline{Command syntax} :
			LOAD filename
			\begin{itemize}
				\item filename : Name of the text file with the extension if it needs one.
			\end{itemize}
			\uline{Error cases}
			\begin{itemize}
				\item Error n°1 : The file is not accessible for reading (maybe it does not exist).
				\item Error n°2 : The file reading generated an exception.
				\item Error n°3 : The program does not understand the file.
			\end{itemize}
			\uline{Output} : OK or ERR and explanation\\
			%\uline{Note} : 

		\subsubsection{Save the current draw}
			\uline{Command syntax} :
			SAVE filename
			\begin{itemize}
				\item filename : Name of the text file with the extension if it needs one.
			\end{itemize}
			\uline{Error cases}
			\begin{itemize}
				\item Error n°1 : The file is not accessible for writing.
				\item Error n°2 : Invalid filename.
			\end{itemize}
			\uline{Output} : OK or ERR and explanation\\
			\uline{Note} : 
			If you enter the name an existent file, it will be overwritten.

		\subsubsection{Delete all objects}
			CLEAR\\
			\uline{Error cases} : No error case.\\
			\uline{Output} : Delete confirmation\\
			\uline{Note} : 
			You can undo a clear command.

		\subsubsection{Exit the application}
			CLEAR\\
			\uline{Error cases} : No error case.\\
			\uline{Output} : No output.\\
			\uline{Note} : 
			If you didn't save your draw, it is lost forever.

		
	\subsection{Save-file format}
		Our saved file begins with : "<B3133drawingFile>".\\
		Next lines describe objects the program needs to recreate to restore the draw.\\
		Action-hictoric is not restored. Objects are memorised as follows :

		\begin{itemize}
			\item \uline{Rectangle} : R Name X1 Y1 X2 Y2
			\item \uline{Segment} : S Name X1 Y1 X2 Y2
			\item \uline{Convex Polygon} : PC Name X1 Y1 X2 Y2 .. Xn Yn
			\item \uline{Reunion Object} : OR Name Name1 .. NameN
			\item \uline{Intersection Object} : OI Name Name1 .. NameN
		\end{itemize}



\end{document}
