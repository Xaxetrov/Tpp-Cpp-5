\documentclass[a4paper, 12pts]{article}

\usepackage[top=3.5cm, bottom=3.5cm, left=3cm, right=3cm]{geometry}

\usepackage[T1]{fontenc}
\usepackage[utf8]{inputenc}
\usepackage[english]{babel}
\usepackage{textcomp}
\usepackage{listings}
\usepackage{authblk} %author tools
\usepackage{enumitem}
\usepackage{amssymb}
\usepackage{graphicx}
\usepackage[normalem]{ulem}

%\usepackage{hyperref} %pour les liens internet

%\usepackage{graphicx} %pour les images
\title{TP4 C++ : Héritage et polymorphisme \\
    \large Conception}
\author{Edern HAUMONT}
\author{Théo THIBAULT}
\affil{B3133}
\date{\today}

%-----------------------------------------------------------------------------------------

\begin{document}

%\begin{titlepage}

\maketitle

%\end{titlepage}

%----------------------------------------------Title end

\section{Introduction}
	This is the document of conception of the application. It is ended by a class diagram which summarize its content.

\section{Objects}
    Objects are instanciated Whe the corresponding form is created.
    \subsection{Point}
    \subsection{Segment}
    \subsection{Polygon}
    \subsection{Rectangle}
    \subsection{Multi-Objects}
    \subsection{Reunion}
    \subsection{Intersection}

\section{Historic of actions}
    To implement the ability for th user to undo and redo actions, we chose a quite simple model :
    \begin{itemize}
        \item When the user calls a valid action, it is executed.
        \item Meanwhile, the command of this action is registered in a list, detailed later.
        \item The program creates an oppostit command which can be executed if the user decides to undo the action.
        This command is stored in another list, but with the same iterator.
        \item Then, when undo and redo are called, we just have to navigate in this list and (re)execute corresponding
        actions
    \end{itemize}

    \subsection{command historic storage}
        Commands are strings which are stored in lists (implemented by a double-linked list in the STL). Their "undo
        counterpart" is stored at the same index but in a different list. We chose to cap the size of both lists to 20,
        which is sufficient for a normal utilisation of the undo possibility. If an action is made but there are
        already 20 commands registered, we erase the oldest to make some space to the new.

\section{Persistance}
    \subsection{Load}
    \subsection{Save}

\end{document}
